\documentclass[11pt]{scrartcl}
\usepackage[utf8x]{inputenc}

\usepackage[T1]{fontenc}
\usepackage{ucs}
\usepackage[ngerman]{babel}
\usepackage{amsmath, amssymb, amstext}
\usepackage{graphicx}
\usepackage{tabularx}

% \usepackage[T1]{fontenc}
% \newcommand{\changefont}[3]{\fontfamily{#1} \fontseries{#2} \fontshape{#3} \selectfont}

\setlength{\textwidth}{460pt}
\title{Formelsammlung elektrische und magnetische Felder}
\author{Ron Gruse}
\date{\today{}, Dresden}

\begin{document}
% \changefont{phv}{m}{n}
\maketitle
\tableofcontents
\newpage
\section{Grundgleichungen}
{\renewcommand{\arraystretch}{1.5}%
\begin{tabularx}{\textwidth}{@{\extracolsep{\fill}}|p{0.25\textwidth}|p{0.35\textwidth}|l|} 
	\hline 
	Formelname & Formel & Einheiten und Größen \\
	\hline
	Coulombsches Gesetz & \(\vec F = k\cdot\frac{Q_1 \cdot Q_2}{|{\vec{r_1} - \vec{r_2}}|^2} \cdot \vec{e}_{\vec{r_1}-\vec{r_2}}\) & [F] = N \\
	& & [Q] = C = A \(\cdot\) s \\
	& \(\vec F = k\cdot Q_1 \cdot Q_2 \cdot \frac{\vec{r_1} - \vec{r_2}}{|{\vec{r_1} - \vec{r_2}}|^3}\) & k \ldots Propotionalitätskonstante \\ 
	& & \hspace{8.5mm}%
	häufig verwendet: \(\frac{1}{4\pi \cdot \epsilon_0}\)\\
	& & \(\vec{r}_n\) \ldots Ortsvektoren \\
	\hline
	% Kontinuierliche \newline Ladungsverteilung & & \\
	Linienladung & \(Q = \int\limits_{C}{} \lambda(\vec{r}) \cdot dr\) & \(\lambda = \frac{dQ}{dr}\), \([\lambda] = \frac{C}{cm}\)\\
	& \(Q = \lambda_0 \cdot l\) für \( \lambda = \lambda_0 = const.\) &  l \ldots Länge von C\\
	Flächenladung & \(Q = \iint\limits_{A}{} \sigma \cdot dA\) & \([\sigma] = \frac{Q}{cm^2}\)\\
	& \(Q = \sigma_0 \cdot A \) für \(\sigma = \sigma_0 = const.\) & \\
	Raumladung & \(Q = \iiint\limits_{V}{} \rho \cdot dV \) & \([\rho] = \frac{Q}{cm^3}\) \\ 
	\hline
\end{tabularx}
\section{Mathematik}
\label{sec:mathematik}
\subsection{Unterstufe}
\label{sec:unterstufe}
\begin{equation*}
	\frac{1}{a} +\frac{1}{b} = \frac{a+b}{ab}
\end{equation*}
\subsection{Oberstufe}
\label{sec:oberstufe}
\begin{equation}
	\label{eq:1}
	\left(\frac{a}{b} \right)' = 0
\end{equation}
\begin{equation}
	\label{eq:2}
	\int\limits_{a}^{b} x^{2} \, dx = \frac{ b^3 -a^3}{3}
\end{equation}
Es gilt die Invariante \(b \neq 0\)\footnote{Nein tut es nicht}
Die Gleichung \eqref{eq:1}

\newpage
\begin{tabularx}{\textwidth}{p{0.25\textwidth}|r}
	Name & Alter \\
	\hline
	Hans & 05    \\
\end{tabularx}

\begin{center}
	blondgelockter Knabe mit kohlrabenschwarzem Haar auf die grüne Bank
	sich setzte, die gelb angestrichen war.
	\vspace{4mm}
	Alice kann es einfach nicht lassen, sie muß dem weißen Kaninchen mit
	der großen Uhr folgen und landet prompt im Wunderland. Auf ihrer Reise
	durch diese fröhlich bunte, aber auch sehr eigenartige Welt begegnet
	sie einer gestiefelten Raupe, dem verrückten Hutmacher und ist zu Gast
	bei einer nicht Geburtstags-Party. Einer hinterlistigen Tigerkatze hat
	es das Mädchen schließlich zu verdanken, daß sie den Zorn der
	Herz-Königin auf sich zieht. So etwas kann einem eigentlich nur im
	Traum passieren.
\end{center}

\end{document}